\usepackage[T1]{fontenc}
\usepackage[utf8]{inputenc}
\usepackage{lmodern}
\usepackage{ae,aecompl}
\usepackage[upright]{fourier}
\usepackage{diagbox}
\usepackage{color}
\usepackage{gensymb}
\usepackage{datetime2}

% Quellenverzeichnis anpassen---------------------------------------
\usepackage[backend=biber,style=apa,  natbib=true,maxbibnames=999,maxcitenames=2]{biblatex}

\usepackage{csquotes}

\usepackage{titlesec}
\titleformat{\paragraph}[hang]{\normalfont\normalsize}{\theparagraph}{1em}{}
\titlespacing*{\paragraph}{0pt}{3.25ex plus 1ex minus .2ex}{1em}
\titleformat{\subparagraph}[hang]{\normalfont\normalsize\itshape}{\thesubparagraph}{1em}{}
\titlespacing*{\subparagraph}{0pt}{3.25ex plus 1ex minus .2ex}{1em}

\usepackage{geometry}
\geometry{top=2.5cm,bottom=2.5cm,left=2.5cm,right=2.5cm}
%normalement 3.5, 3.5, 2.5, 2.5

\setcounter{secnumdepth}{4}
\setcounter{tocdepth}{1}
%\usepackage[toc,page]{appendix}
\usepackage[toc]{appendix}
%\usepackage{natbib}

\usepackage{comment}
\usepackage{amsmath}
\usepackage{amsfonts}
\usepackage{amssymb}
\usepackage{bbm}
\usepackage{nicefrac}
\usepackage[nointegrals]{wasysym}
\usepackage{siunitx}
\usepackage{mathrsfs}
% \usepackage{subfigure}
\usepackage{enumerate}
\usepackage{enumitem}
\usepackage{multicol}
\usepackage[table,xcdraw]{xcolor}
\usepackage{longtable}
\usepackage{lscape}
\usepackage{pdflscape}
\usepackage{graphicx}
\usepackage{subcaption}
\usepackage[figuresleft]{rotating}
\usepackage{tikz}

\usepackage{tabularx}
\usepackage{array,multirow,makecell}


\usepackage{setspace}
\singlespacing
\onehalfspacing

% \usepackage[pdftex]{graphicx}
\usepackage{wrapfig}
\usepackage{here}
\usepackage{float}
\usepackage{multicol}


%\newcommand{\bib}[1]{\up{(\ref{#1}}}

\usepackage{fancyhdr}
\pagestyle{fancy}
\fancyhead{}
\renewcommand{\headrulewidth}{0pt}%Nimmt die Linie im Header weg
% \fancyhead[R]{Institut Geomatik}
% \fancyhead[L]{FHNW}
\fancyfoot[L]{\fontsize{9}{11}\selectfont Bachelor-Thesis am Institut Geomatik der Hochschule für Architektur, Bau und Geomatik FHNW}
\fancyfoot[C]{}
\fancyfoot[R]{\fontsize{9}{11}\selectfont \thepage}
% \fancyfoot[R]{Muttenz - \today}%Fussnote
\renewcommand{\footrulewidth}{0.4pt}
%\setlength{\headheight}{26pt} %ces 3 lignes servent à metre la petite bande en haut (avec le titre de la section), normalement 15pt


\usepackage[bottom]{footmisc}

\usepackage[german]{babel} %Schprache Automatische Test (z.B. Datum, Inhaltsverzeichnis...)
\usepackage[font=footnotesize]{caption}
\usepackage{subcaption}
\usepackage{chngcntr}

\usepackage{lipsum}% just to generate filler text
\usepackage[final]{pdfpages} 

\parindent=0cm

\makeatletter\@addtoreset{section}{part}\makeatother
\renewcommand{\thesection}{\arabic{section}}

%insert pdf page
\usepackage{pdfpages}
\usepackage{pdflscape}

% Matlab code
\usepackage{listings}
\usepackage{color} %red, green, blue, yellow, cyan, magenta, black, white
\definecolor{mygreen}{RGB}{28,172,0} % color values Red, Green, Blue
\definecolor{mylilas}{RGB}{170,55,241}


\lstset{language=Matlab,%
    %basicstyle=\color{red},
    breaklines=true,%
    basicstyle=\ttfamily\footnotesize,%
    morekeywords={matlab2tikz},
    keywordstyle=\color{blue},%
    morekeywords=[2]{1}, keywordstyle=[2]{\color{black}},
    identifierstyle=\color{black},%
    stringstyle=\color{mylilas},
    commentstyle=\color{mygreen},%
    showstringspaces=false,%without this there will be a symbol in the places where there is a space
    numbers=left,%
    numberstyle={\tiny \color{black}},% size of the numbers
    numbersep=9pt, % this defines how far the numbers are from the text
    emph=[1]{for,end,break},emphstyle=[1]\color{red}, %some words to emphasise
    %emph=[2]{word1,word2}, emphstyle=[2]{style},    
}

\usepackage{comment}

\usepackage{todonotes} % ajouter des todo et des missinfigures
%\todo{ceci a faire}
%\missingfigure{la fig doit parler de ca}

% put at end
\usepackage{hyperref} 
\hypersetup{
    colorlinks=true,
    allcolors=black, %link farbe
    %linkcolor=red,
    %citecolor=green,
    %filecolor=magenta,      
    %urlcolor=blue,
    %citebordercolor={0 0 0},
    %filebordercolor={0 0 0},
    %linkbordercolor={0 0 0},
    %runbordercolor={0 0 0},
    %urlbordercolor={0 0 0},
    %pdfborder={0 0 0}
}

\usepackage{tikz}
  \usetikzlibrary{shapes.misc}
\newcommand{\croix}[1][]
{%
\begin{tikzpicture}[baseline=(textbox.base),inner sep=0pt]
\node[cross out,draw,text width=\dimexpr#1] (textbox) {\strut};
\useasboundingbox (textbox);
\end{tikzpicture}%
}

\usepackage{microtype}
\sloppy


\titleformat{\paragraph}[block]{\normalfont\bfseries}{\theparagraph}{1em}{}
\titlespacing*{\paragraph}{0pt}{3.25ex plus 1ex minus .2ex}{0pt}
