\documentclass[11pt,a4paper]{article} % or report for longer documents

\usepackage[T1]{fontenc}
\usepackage[utf8]{inputenc}
\usepackage{lmodern}
\usepackage{ae,aecompl}
\usepackage[upright]{fourier}
\usepackage{diagbox}
\usepackage{color}
\usepackage{gensymb}
\usepackage{datetime2}

% Quellenverzeichnis anpassen---------------------------------------
\usepackage[backend=biber,style=apa,  natbib=true,maxbibnames=999,maxcitenames=2]{biblatex}

\usepackage{csquotes}

\usepackage{titlesec}
\titleformat{\paragraph}[hang]{\normalfont\normalsize}{\theparagraph}{1em}{}
\titlespacing*{\paragraph}{0pt}{3.25ex plus 1ex minus .2ex}{1em}
\titleformat{\subparagraph}[hang]{\normalfont\normalsize\itshape}{\thesubparagraph}{1em}{}
\titlespacing*{\subparagraph}{0pt}{3.25ex plus 1ex minus .2ex}{1em}

\usepackage{geometry}
\geometry{top=2.5cm,bottom=2.5cm,left=2.5cm,right=2.5cm}
%normalement 3.5, 3.5, 2.5, 2.5

\setcounter{secnumdepth}{4}
\setcounter{tocdepth}{1}
%\usepackage[toc,page]{appendix}
\usepackage[toc]{appendix}
%\usepackage{natbib}

\usepackage{comment}
\usepackage{amsmath}
\usepackage{amsfonts}
\usepackage{amssymb}
\usepackage{bbm}
\usepackage{nicefrac}
\usepackage[nointegrals]{wasysym}
\usepackage{siunitx}
\usepackage{mathrsfs}
% \usepackage{subfigure}
\usepackage{enumerate}
\usepackage{enumitem}
\usepackage{multicol}
\usepackage[table,xcdraw]{xcolor}
\usepackage{longtable}
\usepackage{lscape}
\usepackage{pdflscape}
\usepackage{graphicx}
\usepackage{subcaption}
\usepackage[figuresleft]{rotating}
\usepackage{tikz}

\usepackage{tabularx}
\usepackage{array,multirow,makecell}


\usepackage{setspace}
\singlespacing
\onehalfspacing

% \usepackage[pdftex]{graphicx}
\usepackage{wrapfig}
\usepackage{here}
\usepackage{float}
\usepackage{multicol}


%\newcommand{\bib}[1]{\up{(\ref{#1}}}

\usepackage{fancyhdr}
\pagestyle{fancy}
\fancyhead{}
\renewcommand{\headrulewidth}{0pt}%Nimmt die Linie im Header weg
% \fancyhead[R]{Institut Geomatik}
% \fancyhead[L]{FHNW}
\fancyfoot[L]{\fontsize{9}{11}\selectfont Bachelor-Thesis am Institut Geomatik der Hochschule für Architektur, Bau und Geomatik FHNW}
\fancyfoot[C]{}
\fancyfoot[R]{\fontsize{9}{11}\selectfont \thepage}
% \fancyfoot[R]{Muttenz - \today}%Fussnote
\renewcommand{\footrulewidth}{0.4pt}
%\setlength{\headheight}{26pt} %ces 3 lignes servent à metre la petite bande en haut (avec le titre de la section), normalement 15pt


\usepackage[bottom]{footmisc}

\usepackage[german]{babel} %Schprache Automatische Test (z.B. Datum, Inhaltsverzeichnis...)
\usepackage[font=footnotesize]{caption}
\usepackage{subcaption}
\usepackage{chngcntr}

\usepackage{lipsum}% just to generate filler text
\usepackage[final]{pdfpages} 

\parindent=0cm

\makeatletter\@addtoreset{section}{part}\makeatother
\renewcommand{\thesection}{\arabic{section}}

%insert pdf page
\usepackage{pdfpages}
\usepackage{pdflscape}

% Matlab code
\usepackage{listings}
\usepackage{color} %red, green, blue, yellow, cyan, magenta, black, white
\definecolor{mygreen}{RGB}{28,172,0} % color values Red, Green, Blue
\definecolor{mylilas}{RGB}{170,55,241}


\lstset{language=Matlab,%
    %basicstyle=\color{red},
    breaklines=true,%
    basicstyle=\ttfamily\footnotesize,%
    morekeywords={matlab2tikz},
    keywordstyle=\color{blue},%
    morekeywords=[2]{1}, keywordstyle=[2]{\color{black}},
    identifierstyle=\color{black},%
    stringstyle=\color{mylilas},
    commentstyle=\color{mygreen},%
    showstringspaces=false,%without this there will be a symbol in the places where there is a space
    numbers=left,%
    numberstyle={\tiny \color{black}},% size of the numbers
    numbersep=9pt, % this defines how far the numbers are from the text
    emph=[1]{for,end,break},emphstyle=[1]\color{red}, %some words to emphasise
    %emph=[2]{word1,word2}, emphstyle=[2]{style},    
}

\usepackage{comment}

\usepackage{todonotes} % ajouter des todo et des missinfigures
%\todo{ceci a faire}
%\missingfigure{la fig doit parler de ca}

% put at end
\usepackage{hyperref} 
\hypersetup{
    colorlinks=true,
    allcolors=black, %link farbe
    %linkcolor=red,
    %citecolor=green,
    %filecolor=magenta,      
    %urlcolor=blue,
    %citebordercolor={0 0 0},
    %filebordercolor={0 0 0},
    %linkbordercolor={0 0 0},
    %runbordercolor={0 0 0},
    %urlbordercolor={0 0 0},
    %pdfborder={0 0 0}
}

\usepackage{tikz}
  \usetikzlibrary{shapes.misc}
\newcommand{\croix}[1][]
{%
\begin{tikzpicture}[baseline=(textbox.base),inner sep=0pt]
\node[cross out,draw,text width=\dimexpr#1] (textbox) {\strut};
\useasboundingbox (textbox);
\end{tikzpicture}%
}

\usepackage{microtype}
\sloppy


\titleformat{\paragraph}[block]{\normalfont\bfseries}{\theparagraph}{1em}{}
\titlespacing*{\paragraph}{0pt}{3.25ex plus 1ex minus .2ex}{0pt}


% Bibliographie-Datei einbinden
\addbibresource{LiteraturZotero_angepasst.bib}


\setlength {\marginparwidth }{2cm}
\begin{document}

% Seitennumerierung und Kapitelnummerierung in römische Zahlen
\pagenumbering{roman}
\renewcommand{\thesection}{\Roman{section}}



% Titelseite mit eigenem Header und Footer
% Title page
\begin{titlepage}
	\begin{figure}[h]
		\hspace*{-0.9cm}
		\includegraphics[height=1.2cm,draft=false]{Images/fhnw_10mm.png}
	\end{figure} 

    \vspace*{1cm}
    
    \Huge
\begin{flushleft}
    \setstretch{0.9}
    \textbf{Vom Luftbildarchiv zur historischen 3D- Landschaft – Basis für die Beurteilung von Gelände- und Landschaftsveränderungen}

    
    \vspace{0.5cm}
    \Huge
    Bachelor-Thesis 10 / 2025

    
    \vspace{1.5cm}
    
\begin{figure}[H]
    \captionsetup{justification=raggedright, singlelinecheck=false}  % Setzt die Beschriftung auf linksbündig
    \caption*{Ausschnitt der generierten Punktwolke aus den historischen Luftbildern von Muttenz (1953).}
\end{figure}
    
    \vfill
    
    \Large
    \textbf{Sven Uythoven}
    
    \vspace{0.8cm}
    
    \large
    Muttenz, \today
    
    \vspace{0.8cm}
\end{flushleft}

\end{titlepage}


  % Diese Datei enthält die Titelseite mit separatem Header und Footer




% Redlichkeitserklärung, etc.

\textbf{Eine Bachelor-Thesis von: }

\vspace{0.5em}

\textbf{Max Mustermann}\\
AdresseXXX\\
xxx@students.fhnw.ch
\\

\textbf{Eingereicht bei:}

\vspace{0.5em}

\textbf{Prof. Dr. XX}\\ 
XX

\vspace{0.5em}

\textbf{Experte und Betreuer:}

\vspace{0.5em}

\textbf{XX}\\
XX 

\vspace{0.5em}

\textbf{XX}\\
XX 




% Redlichkeitserklärung
 \textbf{Redlichkeitserklärung} \\
Ich versichere, dass ich die vorliegende Arbeit selbstständig und ohne Verwendung anderer als der im Quellenverzeichnis (Literaturverzeichnis) angegebenen Quellen und Hilfsmittel angefertigt habe. Die wörtlich oder inhaltlich den im Literaturverzeichnis aufgeführten Quellen und Hilfsmittel entnommenen Stellen sind in der Arbeit als Zitate kenntlich gemacht und mit einem Verweis auf das Literaturverzeichnis versehen. Ich stimme zu, dass meine Arbeit elektronisch auf Plagiate überprüft werden kann. Diese Bachelor-Thesis ist nicht veröffentlicht, keinen anderen Interessenten zugänglich gemacht und keiner anderen Prüfungsbehörde vorgelegt worden.\\[2cm]
Ort, Datum: Muttenz, 20. Juni 2025
\hspace{2cm}
Max Mustermann: ...........................................

\newpage

% Abstract
\section{Zusammenfassung} 

Diese Arbeit untersucht, ...


\newpage


% Inhaltsverzeichnis (mit Subsection und Subsubsection) - kein Header und Footer
\setcounter{tocdepth}{3}  % Inhaltsverzeichnis mit Sub- und Subsubsection
\tableofcontents


\thispagestyle{fancy}  % Stellt sicher, dass der Header und Footer nach dem Inhaltsverzeichnis wieder angezeigt wird

\newpage

% Seitennumerierung und Kapitelnummerierung auf arabische Zahlen zurücksetzen
\pagenumbering{arabic}
\renewcommand{\thesection}{\arabic{section}}
\setcounter{section}{0} % Zähler zurücksetzen

% Weitere Kapitel und Abschnitte
\section{Einleitung} 
\label{sec:intro}

\begin{enumerate}
    \item Zitierstile kann in Packages.tex unter \textit{ usepackage[backend=biber, style=apa, natbib=true, maxbibnames=999, maxcitenames=2]{biblatex}} angepasst werden. (Linie \textbf{12})

    \item Sprache für automatische Text (z.B. Datum, Abbildung...) kann ebenfalls im Packages.tex angepasst werden, unter \textit{usepackage[german]{babel}} Linie \textbf{89}.

    \item Fusszeile kann in Packages.tex zwischen Linie \textbf{71} und \textbf{82} angepasst werden.
    
    \item Referenz zu ein Kapitel hinzufügen: Kapitel~\ref{sec:intro}

    \item Verwenden von SI-Einheiten: \SI{23}{\centi\meter}
\end{enumerate}






\newpage
\section{Kapitel \& Unterkapitel}


% ----------------------------------------------------

\subsection{Untertitel}
\subsubsection{Unteruntertitel}
\paragraph*{Paragraph}


\section{Abbildungen und Tabellen} \label{Abbildungen und Tabellen}



\subsection{Abbidung}
Abb Referenzieren im Text (\autoref{fig:Luftaufnahme Erdrutschung}).


\begin{figure}[tbhp] % htbp gitb ihm die option "Hier","Top","Bottom","Page" 
    \centering
    \caption{Nadir-Luftaufnahme des Wartenbergs in Muttenz vom 21. März 1953, welche die Auswirkung des Erdrutsch vom 7. April 1952 zeigt.}
    \label{fig:Luftaufnahme Erdrutschung}
\end{figure}

% ---------------------------------------------------------

% Arbeiten mit subfigure:

\begin{figure}[tbhp]
    \centering

    \begin{subfigure}[b]{0.495\textwidth}
        \centering
        \caption{ERDAS: \textit{Position Skip 1, Surface Type Sharp} }
        \label{fig:Profil ERDAS Sharp}
    \end{subfigure}
    \hfill
    \begin{subfigure}[b]{0.495\textwidth}
        \centering
        \caption{ERDAS: \textit{Position Skip 1, Surface Type Shmooth}}
        \label{fig:Profil ERDAS Smooth}
    \end{subfigure}
    
    \vspace{1.5em}

        \begin{subfigure}[b]{0.495\textwidth}
        \centering
        \caption{Metashape: \textit{Quality High, Filtering Mild}}
        \label{fig:Profil Metashape Mild}
    \end{subfigure}
    \hfill
    \begin{subfigure}[b]{0.495\textwidth}
        \centering
        \caption{Metashape: \textit{Quality High, Filtering Agressive}}
        \label{fig:Profil Metashap Agressive}
    \end{subfigure}


    \caption{Vergleich der Punktwolken der Befliegung Gelterkinden 1994 anhand von Profilen. Die Profile zeigen denselben Querschnittsbereich mit einer Breite von \SI{1}{\meter} und dienen zur Beurteilung der Kantenextration an Gebäuden unter Verwendung verschiedener Einstellungen. Die Referenzpunktwolke swissSURFACE3D ist in Schwarz dargestellt. Die jeweils mit ERDAS, Metashape oder PIX4Dmapper generierten Punktwolken sind in Orange abgebildet.}
    \label{fig:vergleich_profile}
\end{figure}

\subsection{Tabelle erstellen}
Tabelle referenzieren im Text \autoref{tab:historische_luftbilder}.




\begin{table}[tbhp]
\centering
\small
\caption{Eigenschaften der verwendeten historischen Luftbilder. Aufgeführt sind sowohl Kameramodel und Bildparameter als auch geometrische Eigenschaften wie Überlappung und Bodenauflösung (GSD).}
\label{tab:historische_luftbilder}
\resizebox{\textwidth}{!}{%
\begin{tabular}{lccccccc}
\hline
\noalign{\vskip 0.8ex}
\textbf{Untersuchungsgebiet} & \textbf{Kameramodel} & \textbf{Überlappung} & \textbf{Bildgrösse} & \textbf{Pixelgrösse} & \textbf{Brennweite} & \textbf{GSD} \\
          \textbf{\& Aufnahmedatum}   &                                     & [\%]               & [\si{\centi\meter}]            &[\si{\micro\meter}]                  & [\si{\milli\meter}]       & [\si{\centi\meter}]                 \\
\noalign{\vskip 0.8ex}
\hline
\noalign{\vskip 0.8ex}
\textbf{Muttenz} \\ 

 21. Mai 1953 & Wild RC5    & 60 & 18 × 18   & 21   & 210.14 & 39 \\
\hline
\noalign{\vskip 0.8ex}

\textbf{Gelterkinden} \\
 24. Mai 1953 & Wild RC5    & 60 & 18 × 18   & 21   & 210.14 & 39 \\
 7. November 1994 & Wild RC30 & 70 & 23 × 23   & 14  & 152.52 & 38 \\

\hline
\end{tabular}%
}
\end{table}




\newpage
\section{Zitieren}


Im Text zitieren: \textcite{medrzycka_rapid_2023}

In Klammern zitieren:  \parencite{wiggenhagen_taschenbuch_2021}


\newpage
\section{Fazit}


Anhand der erzeugten Punktwolken und...














\newpage

% \phantomsection
% \addcontentsline{toc}{section}{Abbildungsverzeichnis}
% \listoffigures

% \phantomsection
% \addcontentsline{toc}{section}{Tabellenverzeichnis}
% \listoftables

\newpage
% \printbibliography[heading=bibintoc, title={Quellen- und Hilfsmittelverzeichnis}]
\section{Quellen- und Hilfsmittelverzeichnis}
\subsection*{Quellenverzeichnis}

\printbibliography[heading=none]





\newpage

\section*{Abkuerzungsverzeichnis}
\addcontentsline{toc}{section}{Abkuerzungsverzeichnis}

\medskip

\begin{tabular}{ll}
\textbf{Abkuerzung} & \textbf{Bedeutung} \\
\hline
C2C & Cloud to Cloud \\
CP & Check Point (Kontrollpunkt) \\
DPC & Dichte Punktwolke \\
DOM & Digitales Oberflaechenmodell \\
\end{tabular}

\newpage


%\newpage
%\nocite{*}
%\printbibliography[title={References},heading=bibintoc]

\appendix
%\counterwithin{figure}{section}
%\counterwithin{table}{section}
\appendix


\section{Python-Skript: Bild\_download.py}
\label{anhang:Skript_Bild_download.py}



\newpage
% ----------------------------

\newpage

\section{Korrelationsmatrix Kamerakalibrierung}
\label{anhang:Ergebnisse_Kamerakalibrierung}
\subsection{Kamerakalibrierung mit F, Cx, Cy, K1-3 und P1-2}
\label{anhang:Kalibrierung_Muttenz_ohne_B1_B2_K4}
\begin{table}[htbp]
    \centering
    \caption*{Kalibrierungsparameter und Korrelationsmatrix der Kamerakalibrierung mit den Parametern F, Cx, Cy, K1–3 und P1–2, berechnet in Agisoft Metashape. Die Tabelle zeigt die berechneten Werte, deren geschätzte Genauigkeiten sowie die Korrelationsmatrix der Parameter.}
    \setlength{\tabcolsep}{4.5pt}
    \renewcommand{\arraystretch}{1.2}
    \begin{tabular}{|l|l|l|r|r|r|r|r|r|r|r|}
    \hline
    & \textbf{Wert} & \textbf{Genauigkeiten}  & \textbf{F} & \textbf{Cx} & \textbf{Cy} & \textbf{K1} & \textbf{K2} & \textbf{K3} & \textbf{P1} & \textbf{P2} \\
    \hline
    \textbf{F}   & 209.127      & 0.28  & 1.00 & 0.08 & -0.08 & 0.61 & -0.20 & 0.19 & -0.04 & -0.18 \\
    \textbf{Cx}  & 2.065      & 0.13  &      & 1.00 &-0.16 & -0.13 & 0.08 & -0.07 & 0.69 & -0.22 \\
    \textbf{Cy}  & -0.534    & 0.10  &      &      & 1.00 & 0.07 & -0.05 & 0.03 & -0.04 & 0.58 \\
    \textbf{K1}  & 1.420 × 10\textsuperscript{-2}   & 0.64 × 10\textsuperscript{-3} &     &      &      & 1.00 & -0.75 & 0.71 & -0.16 & -0.08 \\
    \textbf{K2}  & -8.054 × 10\textsuperscript{-2}  & 3.3 × 10\textsuperscript{-3}  &     &      &      &      & 1.00 & -0.99 & 0.09 & 0.02 \\
    \textbf{K3}  & 0.125     & 8.1 × 10\textsuperscript{-3}&     &      &      &      &      & 1.00 & -0.07 & -0.02 \\
    \textbf{P1}  & 8.22 × 10\textsuperscript{-4}   & 9.2 × 10\textsuperscript{-5} &     &      &      &      &      &      & 1.00 & -0.01 \\
    \textbf{P2}  & -0.001  & 0.11 × 10\textsuperscript{-3} &    &      &      &      &      &      &      & 1.00 \\
    \hline
    \end{tabular}
\end{table}

\subsection{Kamerakalibrierung mit F, Cx, Cy, K1-3}
\label{anhang:Kalibrierung_Muttenz_ohne_B1_B2_P1_P2_K4}
\begin{table}[htbp]
    \centering
    \caption*{Kalibrierungsparameter und Korrelationsmatrix der Kamerakalibrierung mit den Parametern F, Cx, Cy, K1–3, berechnet in Agisoft Metashape. Die Tabelle zeigt die berechneten Werte, deren geschätzte Genauigkeiten sowie die Korrelationsmatrix der Parameter.}
    \setlength{\tabcolsep}{4.5pt}
    \renewcommand{\arraystretch}{1.2}
    \begin{tabular}{|l|l|l|r|r|r|r|r|r|}
    \hline
    & \textbf{Wert} & \textbf{Genauigkeiten}  & \textbf{F} & \textbf{Cx} & \textbf{Cy} & \textbf{K1} & \textbf{K2} & \textbf{K3} \\
    \hline
    \textbf{F}   & 208.838      & 0.27  & 1.00 & 0.11 & 0.03 & 0.60 & -0.19 & 0.18 \\
    \textbf{Cx}  & 0.996    & 0.091 &      & 1.00 & -0.01 & -0.04 & 0.03 & -0.02 \\
    \textbf{Cy}  & 0.099    & 0.082 &      &      & 1.00 & 0.14 & -0.07 & 0.05 \\
    \textbf{K1}  & 1.455 × 10\textsuperscript{-2}   & 0.64 × 10\textsuperscript{-3} &     &      &      & 1.00 & -0.75 & 0.71 \\
    \textbf{K2}  & -8.016 × 10\textsuperscript{-2}  & 3.4 × 10\textsuperscript{-3}  &     &      &      &      & 1.00 & -0.99 \\
    \textbf{K3}  & 0.122      & 8.5 × 10\textsuperscript{-3}&     &      &      &      &      & 1.00 \\
    \hline
    \end{tabular}
\end{table}



\newpage



\end{document}
